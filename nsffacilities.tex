\required{Facilities, Equipment, and Other Resources}

\setcounter{page}{1}
\renewcommand{\thepage}{Facilities, Equipment, and Other Resources - Page \arabic{page} of 2}

\begin{center}
\textbf{\large Facilities, Equipment \& Other Resources}
\end{center}

%	Identify the facilities to be used, as appropriate, indicate their capacities,
%	pertinent capabilities, relative proximity, and extent of availability to the project.
%	Use ``Other" to describe the facilities at any other performance sites listed and at
%	sites for field studies.

% \textbf{Laboratory:}
% 
% \textbf{Clinical:}
% 
% \textbf{Animal:}

%\textbf{Computer:}

%	List in detail the computing capabilities of the university and the department

%\textbf{Office:}

%\textbf{Other:}

%\textbf{MAJOR EQUIPMENT:}

%	List the most important items available for this project and, as appropriate
%	identifying the location and pertinent capabilities of each.

%\textbf{OTHER RESOURCES:}

%	Provide any information describing the other resources available for the
%	project. Identify support services such as consultant, secretarial,
%	machine shop, and electronics shop, and the extent to which they will be
%	available for the project. Include an explanation of any consortium/
%	contractual arrangements with other organizations.

\textbf{UC Davis}

Dr. Ross-Ibarra has four standard laboratory benches as part of a shared lab space at UCD.  The shared space is the single largest lab space on campus, and provides for seamless interaction between the labs housed there.  The space currently houses three other PIs, all working on the genetics and genomics of economically important plant taxa (Dubcovsky, Neale, Dandekar). The lab is equipped with standard equipment and tools for molecular biology, including freezers and refrigeration, a shared liquid handling robot, thermal cyclers, centrifuges, gel rigs, balances, and standard molecular biology supplies.  A dedicated low-humidity refrigerator for seed storage is available through the university, and low-humidity storage cabinets for tissues and temporary seed storage are in the laboratory. Dr. Ross-Ibarra occupies half of a large office suite that includes a conference room and cubicle space for 25 people.  Both Macintosh and PC workstations are available for student and postdoc employees. The PI is a contributing partner in a large computer cluster, giving the lab dedicated access to 192 processors, with the opportunity for use of nearly 800 additional CPU as resources allow. Recent (2013) additions to the cluster have provided it with additional CPU as well as six new shared high-memory (512Gb RAM) nodes, one of which is dedicated to the Ross-Ibarra lab. Dr. Ross-Ibarra is a faculty member of the UC Davis Genome Center, a large facility that includes bioinformatics, genotyping, metabolomics, proteomics, and expression analysis cores able to perform a variety of genomics analyses at cost for UC Davis faculty. The Genome Center also rents time on its equipment, including a bioanlyzer and library preparation robots. As a member of the Genome Center, Dr. Ross-Ibarra also has access to their additional computational facilities. UC Davis has also entered into a recent partnership with BGI (formerly the Beijing Genomics Institute) to provide additional high-throughput sequencing services via a new Sacramento-based sequencing facility.

Dr. Coop's dry space is located on the 3rd floor of the Storer building, which houses the Department of Evolution and Ecology. The space is newly renovated space and consists of 3 offices that can seat a 8 total of people, and a conference room. In addition members of the lab have access to an additional conference room and other offices shared with the Begun, Langley, Lott, Kopp and Turelli groups. This group is part of the larger Center and Graduate Group for Population Biology, one of the leading graduate training programs in ecology and evolution in the world. Each current member of Dr. Coop’s group has a quad-core Mac pro. The computers are loaded with all the necessary software (Word, R, Mathematica etc.) and are connected to the university network as well as to color and black and white printers. The Coop lab has access to the genome center computational facilities: http://www.genomecenter.ucdavis.edu/core-facilities/. 

\textbf{Iowa State}

Project components completed in the Hufford Laboratory will include mapping population development, DNA isolation and PCR, and population genetic analysis of genotyping data. Population development will be carried out in field space available at the Curtiss Farm of Iowa State University (ISU).  This facility is equipped with irrigation, tractors, tillage equipment, planters, and combines.  Seed processing and cold storage facilities are also available on the ISU campus.  The Hufford Laboratory has all equipment necessary for DNA isolation and PCR including centrifuges, thermal cyclers, an ultra-low freezer, water baths, a pH meter, balances, and an electrophoresis system. A gel imaging system and a NanoDrop spectrophotometer for DNA quantification are accessible through the Center for Plant Responses to Environmental Stresses at ISU. The DNA Facility at ISU provides access to cutting-edge genomic technology including HiSeq and MiSeq Illumina sequencing and library preparation for both paired-end and mate-pair approaches.  Data analyses will be carried out using the High Performance Computing clusters available at ISU. Dr. Hufford currently has access to the Lightning3 cluster which has a mix of Opteron based servers, consisting of 18 SuperMicro servers with core counts ranging from 32 to 64 and 256 to 512 GB of memory.

\textbf{USDA-ARS, Missouri}

Dr. Flint-Garcia has 600 sq. ft of laboratory space in Curtis Hall, on the University of Missouri campus.  The laboratory is fully equipped for molecular genetics, including a chemical hood, a Beckman table top centrifuge with multiple tube buckets, a Tetrad four plate thermalcycler, several freezers, ultra-low freezers and refrigerators, water baths, a pH meter, and balances.  In the building, laboratory personnel have ready access to ultracentrifuges and rotors, growth chambers, an autoclave, lyophilizers, a Sorvall high speed preparative centrifuge with four rotors, a shaker-incubator for bacterial cultures, a chromatography cabinet, electrophoresis equipment for DNA, RNA protein and DNA sequence analysis, a plate reading spectrophotometer/flourometer, a pulse-field electrophoresis system, six Thermolyne thermalcyclers, and four Tetrad four plate thermalcyclers. Dr. Flint-Garcia has multiple personal computers, and computing resources including weekly data backups, direct access to a Sun Ultra10 Unix Workstation and NT server for data sharing, and IT support from USDA-ARS.  In addition, the co-PI has access to the Lewis bioinformatics cluster (over 180 compute nodes with more than 1200 processor cores and 5400 GB of memory) via the University of Missouri Bioinformatics Core Facility. Dr. Flint-Garcia has 120 sq. ft of office space and ample office and desk space for postdocs, technicians and graduate students. Dr. Flint-Garcia shares two ABI 3100 DNA sequencers, an ABI 7900HT RTPCR machine, and a Beckman NxP robot used primarily for DNA extractions with Mel Oliver and Mike McMullen, and other USDA scientists in the unit. Dr. Flint-Garcia has access to greenhouse and field space (with irrigation capability; University of Missouri South Farm and Bradford Research Center), seed processing and cold storage space, and use of winter nursery facilities in Puerto Rico. The co-PI has access to a complete set of field equipment including multiple tractors, tillage equipment, a 4-row plot planter, and a 2-row plot combine. 

\textbf{LANGEBIO} 

Langebio's mandate is to conduct top-ranked research while promoting genomic knowledge for the protection and sustainable use of Mexican biodiversity. Its unique location in the agricultural center of Mexico facilitates field sampling and field experimentation. We have ample experience growing maize in nurseries located on the West Coast (Valle de Banderas, Nayarit), in Central Mexico (Irapuato; Celaya, Guanajuato), and have begun to establish additional sites in the high valleys of Central Mexico (Queretero; Estado de Mexico). We regularly conduct field expeditions to collect plants in both the dry regions of Northern Mexico (maize collections in Chihuahua, Lamiaceae throughout the Northeast) and the lower valleys of the Eje Volcanico and Costa del Pacifico (Teocintle and maize, Solanaceae, and Cucurbitaceae). Research at Langebio is supported by greenhouse facilities and two service units: Genomics and Mass Spectrometry, both of them equipped with state-of-the-art instrumentation, including several next-generation sequencing machines and diverse mass spectrometry equipments. Other facilities include a computation cluster and a specialized clean room for ancient DNA analysis.
