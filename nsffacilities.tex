\required{Facilities, Equipment, and Other Resources}

\setcounter{page}{1}
\renewcommand{\thepage}{Facilities, Equipment, and Other Resources - Page \arabic{page} of 2}

%\begin{center}
%\textbf{\large Facilities, Equipment \& Other Resources}
%\end{center}

%	Identify the facilities to be used, as appropriate, indicate their capacities,
%	pertinent capabilities, relative proximity, and extent of availability to the project.
%	Use ``Other" to describe the facilities at any other performance sites listed and at
%	sites for field studies.

% \textbf{Laboratory:}
% 
% \textbf{Clinical:}
% 
% \textbf{Animal:}

%\textbf{Computer:}

%	List in detail the computing capabilities of the university and the department

%\textbf{Office:}

%\textbf{Other:}

%\textbf{MAJOR EQUIPMENT:}

%	List the most important items available for this project and, as appropriate
%	identifying the location and pertinent capabilities of each.

%\textbf{OTHER RESOURCES:}

%	Provide any information describing the other resources available for the
%	project. Identify support services such as consultant, secretarial,
%	machine shop, and electronics shop, and the extent to which they will be
%	available for the project. Include an explanation of any consortium/
%	contractual arrangements with other organizations.

\paragraph{\textbf{Iowa State University}}\

Project components completed in the Hufford Laboratory at Iowa State University (ISU) will include DNA isolation and preparation for genotyping and population genetic analysis of genotyping and full-genome sequence data. The Hufford Laboratory has all equipment necessary for DNA isolations, quality control and preparation for genotyping including centrifuges, thermal cyclers, an ultra-low freezer, water baths, a pH meter, balances, and an electrophoresis system. A gel imaging system and a NanoDrop spectrophotometer for DNA quantification are accessible through the Center for Plant Responses to Environmental Stresses at ISU. Genotyping will be carried out using a reduced representation approach to next-generation sequencing known as Genotyping by Sequencing (GBS) at the Genomic Diversity Facility at Cornell University (see letter of commitment from Sharon Mitchell). Full-genome sequencing will be carried out at the DNA Facility at ISU, which provides access to cutting-edge genomic technologies including HiSeq and MiSeq Illumina sequencing and library preparation for both paired-end and mate-pair approaches. Data analyses will be carried out using the High Performance Computing clusters available at ISU. Dr. Hufford currently has access to the Lightning3 cluster which has a mix of Opteron based servers, consisting of 18 SuperMicro servers with core counts ranging from 32 to 64 and 256 to 512 GB of memory.  Broader impacts at ISU will be facilitated by the \emph{Symbi} program. \emph{Symbi} was Iowa's first GK12 program and represents a partnership between the Des Moines Public School System and Iowa State University.  Staff members from \emph{Symbi} have experience facilitating over 30 previous graduate student fellows in communicating their science to grade school students and will assist the graduate student funded by this project to do the same (see letter of commitment from \emph{Symbi}).

\paragraph{\textbf{UC Davis}}\

Dr. Ross-Ibarra has four standard laboratory benches as part of a shared lab space at UCD.  The shared space is the single largest lab space on campus, and provides for seamless interaction between the labs housed there.  The space currently houses three other PIs, all working on the genetics and genomics of economically important plant taxa (Dubcovsky, Neale, Dandekar). The lab is equipped with standard equipment and tools for molecular biology, including freezers and refrigeration, a shared liquid handling robot, thermal cyclers, centrifuges, gel rigs, balances, and standard molecular biology supplies.  A dedicated low-humidity refrigerator for seed storage is available through the university, and low-humidity storage cabinets for tissues and temporary seed storage are in the laboratory. Dr. Ross-Ibarra occupies half of a large office suite that includes a conference room and cubicle space for 25 people.  Both Macintosh and PC workstations are available for student and postdoc employees. Dr. Ross-Ibarra is a contributing partner in a large computer cluster, giving the lab dedicated access to 192 processors, with the opportunity for use of nearly 800 additional CPU as resources allow. Recent (2013) additions to the cluster have provided it with additional CPU as well as six new shared high-memory (512Gb RAM) nodes, one of which is dedicated to the Ross-Ibarra lab. Dr. Ross-Ibarra is a faculty member of the UC Davis Genome Center, a large facility that includes bioinformatics, genotyping, metabolomics, proteomics, and expression analysis cores able to perform a variety of genomics analyses at cost for UC Davis faculty. The Genome Center also rents time on its equipment, including a bioanlyzer and library preparation robots. As a member of the Genome Center, Dr. Ross-Ibarra also has access to their additional computational facilities. UC Davis has also entered into a recent partnership with BGI (formerly the Beijing Genomics Institute) to provide additional high-throughput sequencing services via a new Sacramento-based sequencing facility.

\paragraph{\textbf{Partners in Mexico and Guatemala}}\

Senior Personnel on this project include Luis Eguiarte of UNAM in Mexico City and Claudia Calder\'{o}n, a Guatemalan national.  This project will benefit greatly from their many years of experience working in the field in Mexico and Guatemala respectively.  In addition we have confirmed commitments from Ruairidh Sawers of Langebio in Irapuato, Mexico and Salvador Montes-Hernandez of Inifap in Celaya, Mexico (see attached commitment letters) to assist with common garden experiments.  Between Dr. Sawers and Dr. Montes-Hernandez, our collaborators have ample experience growing maize and teosinte in nurseries located on the West Coast (Valle de Banderas, Nayarit), in Central Mexico (Irapuato and Celaya, Guanajuato), and in the high valleys of Central Mexico (Queretero, Estado de Mexico). They also regularly conduct field expeditions to collect plants in both the dry regions of Northern Mexico (maize collections in Chihuahua, Lamiaceae throughout the Northeast) and the lower valleys of the Eje Volcanico and Costa del Pacifico (teosinte and maize, Solanaceae, and Cucurbitaceae).  A commitment has also been confirmed (see attached letter) from Mario Fuentes L\'{o}pez to assist with teosinte collection in Guatemala.
