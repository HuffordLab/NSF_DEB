
%%%%%%%%% PROJECT SUMMARY -- 1 page, third person
% e.g:  "The PI will prove" not "I will prove"

%Below are the pagination, font size, spacing and margin
%instructions for NSF proposals: \\
%
%FastLane does not automatically paginate a proposal.
%Each section of the proposal must be individually
%paginated prior to upload to the system. \\
%
%Use Computer Modern family of fonts at a font size of 11 points or
%larger. A font size of less than 10 points may be used for mathematical
%formulas or equations, figure, table or diagram captions and when
%using a Symbol font to insert Greek letters or special characters.
%The text must still be readable. The use of small type not in compliance with the NSF guidelines
%may be grounds for NSF to return the proposal without review. \\
%
%No more than 6 lines of text within a vertical space of 1 inch. \\
%
%Margins, in all directions, must be at least an inch. \\
%
%
\required{Project Summary}

\paragraph{Overview} The proposed project will investigate the genome-wide effects of hybridization and introgression in the genus \emph{Zea}.  First, investigators will study  incipient speciation between the lowland-adapted \emph{Z. mays} ssp. \emph{parviglumis} (hereafter, \emph{parviglumis}) and the highland-adapted \emph{Z. mays} ssp. \emph{mexicana} (hereafter, \emph{mexicana}).  Through field collections, genotyping and common garden studies, the investigators will assess what fraction of the genome is porous to gene flow in hybrid zones, how fitness of these taxa varies across a hybrid zone, and how allele frequency clines are patterned at adaptive loci. Second, investigators will determine the impact of hybridization and introgression between domesticated maize (\emph{Z. mays} ssp. \emph{mays}) and wild  \emph{Zea}. Population genomic analyses of sympatric collections will be used to assess whether maize served as a bridge for gene flow between otherwise allopatric \emph{Zea} species and whether maize received gene flow from wild relatives that facilitated its adaptation to new environments.

% This should be a brief statement of the problem you plan to address.
% It should look something like an abstract. 

%The project summary should be a description of the proposed activity suitable
%for publication, no more than one page in length. It should not be
%an abstract of the proposal, but rather a self-contained description of
%the activity that would result if the proposal were funded. The summary
%should be written in the third person and include a statement of objectives
%and methods to be employed. It should be informative to other persons
%working in the same or related fields and understandable to a scientifically
%or technically literate lay reader. \\
%
%The summary must clearly address in separate statements (within the one-page summary):
%the intellectual merit of the proposed activity; and the broader impacts
%resulting from the proposed activity. Proposals that do not separately
%address both criteria within the one-page Project Summary will be returned without
%review. \\

\paragraph{Intellectual Merit}  Much progress has been made in the study of hybridization and introgression through the development of theory, through field-based ecological research, and through genetic analyses based on a limited number of molecular markers. However, much remains to be discovered regarding how these evolutionary processes have shaped genomes. The research proposed here will leverage the genomic resources of the maize model system to investigate how hybridization and introgression have molded the genomes of both wild \emph{Zea} species and domesticated maize on two different timescales: 1) An evolutionary timescale covering 60,000 generations of divergence between \emph{parviglumis} and \emph{mexicana}; and 2) An ecological timescale in which maize has spread across the Americas and adapted to local conditions. The analysis on an evolutionary timescale will generate basic knowledge on the process of incipient speciation and the porous nature of the genomes of diverging species, whereas the analysis on an ecological timescale can inform, for example, the study of biological invasions and the role of introgression in facilitating rapid local adaptation.

% This is why your project is interesting and will help further
% knowledge in the field of mathematics. 

%How important is the proposed activity to advancing
%knowledge and understanding within its own field or across different fields?
%How well qualified is the proposer (individual or team) to conduct the project?
%(If appropriate, the reviewer will comment on the quality of prior work.)
%To what extent does the proposed activity suggest and explore creative, original,
%or potentially transformative concepts? How well conceived and organized is the
%proposed activity? Is there sufficient access to resources?  \\

\paragraph{Broader Impacts}

The investigators will achieve societally relevant outcomes in the proposed project by providing STEM training opportunities for undergraduate and graduate students, participating in Iowa State University�s GK12 Fellowship program, and establishing an exchange program between universities in the United States and Mexico. This project will provide ample training opportunities for both undergraduate and graduate students in laboratory, computational, and field-based research. The investigators have successfully recruited minority students into their research programs in the past and will make every effort to do so as part of this proposed work. The graduate student funded to work at Iowa State University will participate in the University's GK12 program that serves the Des Moines public school system. Through this opportunity, the graduate student will bring their research on maize evolution into the classrooms of the most diverse public school system in the state of Iowa. The \emph{Zea} study system provides an excellent opportunity to deliver evolution training to middle and high school students in a state dominated by maize agriculture. Finally, the proposed exchange program would create an opportunity for students from the United States to conduct research internationally and allow these students to interact with visiting students from Mexico. Through these interactions, students will be better prepared for modern STEM research, which is often highly collaborative and international in nature.

% There are 4 kinds of broader impacts.
% 1. advance discovery and understanding while promoting teaching,
% training and learning
% 2. broaden the participation of underrepresented groups
% 3. disseminated broadly to enhance scientific and technological
% understanding
% 4. benefits of the proposed activity to society

