
\documentclass[11pt,letterpaper]{article}

%%%%%%%%%%%%%%%%%%%%%%%%%%%%%%%%%%%%%%%%%%%%%%%%%%%%%%%%%%%%%%%%%%%%%%%%%
\pagestyle{plain}                                                      %%
%%%%%%%%%% EXACT 1in MARGINS %%%%%%%                                   %%
\setlength{\textwidth}{6.5in}     %%                                   %%
\setlength{\oddsidemargin}{0in}   %% (It is recommended that you       %%
\setlength{\evensidemargin}{0in}  %%  not change these parameters,     %%
\setlength{\textheight}{8.5in}    %%  at the risk of having your       %%
\setlength{\topmargin}{0in}       %%  proposal dismissed on the basis  %%
\setlength{\headheight}{0in}      %%  of incorrect formatting!!!)      %%
\setlength{\headsep}{0in}         %%                                   %%
\setlength{\footskip}{.5in}       %%                                   %%
%%%%%%%%%%%%%%%%%%%%%%%%%%%%%%%%%%%%                                   %%
\newcommand{\required}[1]{\section*{\hfil #1\hfil}}                    %%
\renewcommand{\refname}{\hfil References Cited\hfil}                   %%
\bibliographystyle{plainnat}                                              %%
%%%%%%%%%%%%%%%%%%%%%%%%%%%%%%%%%%%%%%%%%%%%%%%%%%%%%%%%%%%%%%%%%%%%%%%%%

%PUT YOUR MACROS HERE
\usepackage{hyperref} %for urls
\usepackage{graphicx} %for includegraphics
\usepackage[table]{xcolor} %for table line color alteration
\usepackage{enumitem}  %fancy enumerated lists
\usepackage{wrapfig} %wrap text around figures
\usepackage[leftcaption]{sidecap} % side captions
\sidecaptionvpos{figure}{c} %position side caption

\usepackage[round,authoryear]{natbib} %biblio format!
\usepackage{multirow} %multirow tables
\usepackage{booktabs} %funky tabs in tables
\renewcommand{\thepage}{f. Budget Justification - Page \arabic{page} of 2}

%\includeonly{NSFsumm}

\begin{document}
\setcounter{page}{1}

\required{Budget Justification}
%\begin{center}
%\emph{Maximum of 3 pages}
%\end{center}

\subsection*{Personnel}

Two months of funding per year are requested for Dr. Claudia Irene at \$4,425 per month as an Asst. Project Scientist. Dr. Irene will work with Dr. Ross-Ibarra's group to collect and help to prepare and analyze teosinte samples, and will participate in outreach activities.  She will visit UC Davis during the course of the grant for bioinformatic training and to collaborate on analysis, and will have regular meetings via Skype with Dr. Ross-Ibarra during the course of the grant.

\subsection*{Other Personnel}
\paragraph{Postdoctoral Scholar}
Funds are requested to support one full-time postdoctoral scholar for the duration of the proposal. The postdoc will have primary responsibility for data analysis and writing publications, but will also contribute to sample preparation. The salary for this position begins at \$42,000, and increases 5\% each year thereafter.

%\paragraph{Graduate students} 
%Funds are requested to support two graduate students each for 6 months during the academic year for each year of the project. At UC Davis, the current pay rate for doctoral students at 50\% FTE is \$27,319 during the academic year. Included is the estimated annual salary increase of 3\%.  The two students will be working on analysis of GBS data in the introgression and admix population genetic sections of \ref{sec:popgen}, and will likely help with QTL analysis and sequencing in \ref{sec:qtl}, and potentially RNA-seq analysis in \ref{sec:funchar}.

%\paragraph{Technician}
%Funds are requested for the first three years of the grant for a 50\%-time technician (Laboratory Assistant III) to extract DNA and RNA, prepare genomic and transcriptomic sequencing libraries, and perform root chilling experiments.  The salary for this positions is set at \$36,000 (\$18,000 for 50\% time), with an annual increase of 5\%.

\subsection*{Fringe Benefits}
Fringe benefits are applied to personnel salaries using the university approved rates:
\begin{itemize}
\item Sr. Personnel - 10.9\% in FY 2015, 11.4\% in FY 2016, and 11.7\% in FY 2017.
\item Postdocs - 17\% in FY 2015, 18\% in 2016, and 19\% in 2017
%\item Graduate students - 1.3\% for all years.
%\item Undergraduate students - \% in FYs 2012, 2013, and 2014
%\item Technician - 50.4\%(1/1/2015-6/31/2015), 53.4\%(6/31/2015-6/31/2016), 55.7\%(6/31/2016-6/31/2017), 57.3\%(6/31/2017-12/31/2017)
%\item Part time staff - \% in FYs 2012, 2013, and 2014
\end{itemize}

\subsection*{Equipment}

No equipment funds are requested.

\subsection*{Travel}

Travel for two travelers (the PI or Sr. Personnel and the postdoc) for domestic or international conference travel is budgeted each year at \$3,000.  Travel for Sr. Personnel Dr. Irene to Iowa State and/or Davis for a bioinformatics training and participation in the outreach program at ISU is budgeted at \$1,000 each year.

Travel for Senior Personnel Dr. Irene to collect in Guatemala in year 1 of the grant is budgeted at \$2,000.

\subsection*{Other Direct Costs}

 \paragraph{Materials and Supplies}
In each of the three years of the grant, \$5,000 is requested in materials and supplies for library prep for whole genome sequencing, and DNA extraction and preparation for GBS.  This also includes funds for standard office supplies, computer supplies (extra storage for our cluster, backup drives for lab members), and other miscellaneous expenses. 

\paragraph{Whole genome sequencing}
The genomes of each of four teosinte will be resequenced to a depth of 20-30X using 2 lanes of paired end 150bp reads on an Illumina HiSeq 2500. Current lane costs are approximately \$2,932 per lane, and library preparations costs are approximately \$75, for a total cost of \$11,878.00 for two \emph{Zea mays} ssp. \emph{mexicana} in year 1 and \$15,787 for the larger \emph{Zea luxurians} genome in year 2.

\paragraph{GBS}
Genotyping-by-sequencing will be performed for our introgression analyses admixture population genetic analyses. GBS will be performed at the Institute for Genomic Diversity at Cornell.  Current prices are \$45 per sample to run samples at 48-plex.  We will genotype 288 individuals in year 1 for a cost of \$12,960, and 120 individuals in year 2 for a cost of \$5,400. 

%\paragraph{RNA sequencing}
%In total, RNA sequencing will be performed on 384 individuals (8 inbreds x 2 stages x 2 tissues x 2 environments x 3 replicates + 8 NILs x 2 genotypes x 2 stages x 2 environs x 3 replicates).  Cost to prepare RNA libraries in our lab are approximately \$100 per library, and sequencing costs for single-end 50bp reads at the UCD Genome Center are approximately \$1,000 per lane.  Multiplexing 12 barcodes per lane, this comes out to 32 lanes of sequence and a total cost of \$70,400.

%\paragraph{Field fees}
%Fees for the field experiments in our highland and lowland field sites (Table \ref{tab:locales}) are approximately \$60,000 the first three years of the experiment to allow development of the mapping populations and two replicates of the phenotyping.  These fees include land rental and basic management (planting, watering, weeding, fertilizing), as well as station fees to hire manual labor for phenotyping.  These fees decrease to \$10,000 in the last two years of the proposal as subsequent field experiments including evaluation of NILs and RNA-seq lines, will be considerably smaller. Field fees total \$200,000 across the five years of the grant.

%\paragraph{Graduate Student Tuition}
%Tuition for graduate students is charged to the project in proportion to the amount of effort the graduate student will work on the project. For a graduate student employed on the project for 9 academic months at 50\% FTE, the tuition charge is \$31,546 in FY 2015 to account for out-of-state tuition, \$17,266 in FY 2016 and increasing 5\% each subsequent year.

\paragraph{Publication Costs}
In year two and three \$1,600 is requested for publication fees to an open access journal. 

\subsection*{Total Direct Costs}

Total direct costs for UCD come to \$264,065.  

\subsection*{Indirect Costs}
Indirect costs are calculated on Total Direct Costs using F\&A rates approved by US Department of Health and Human Services. For this project, F\&A rates of 56.5\% were used from July 1, 2015 through June 30, 2016, and 57\% from July 1, 2016 until the end of the project.

\end{document}

