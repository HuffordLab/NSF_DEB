
\documentclass[11pt,letterpaper]{article}

%%%%%%%%%%%%%%%%%%%%%%%%%%%%%%%%%%%%%%%%%%%%%%%%%%%%%%%%%%%%%%%%%%%%%%%%%
\pagestyle{plain}                                                      %%
%%%%%%%%%% EXACT 1in MARGINS %%%%%%%                                   %%
\setlength{\textwidth}{6.5in}     %%                                   %%
\setlength{\oddsidemargin}{0in}   %% (It is recommended that you       %%
\setlength{\evensidemargin}{0in}  %%  not change these parameters,     %%
\setlength{\textheight}{8.5in}    %%  at the risk of having your       %%
\setlength{\topmargin}{0in}       %%  proposal dismissed on the basis  %%
\setlength{\headheight}{0in}      %%  of incorrect formatting!!!)      %%
\setlength{\headsep}{0in}         %%                                   %%
\setlength{\footskip}{.5in}       %%                                   %%
%%%%%%%%%%%%%%%%%%%%%%%%%%%%%%%%%%%%                                   %%
\newcommand{\required}[1]{\section*{\hfil #1\hfil}}                    %%
\renewcommand{\refname}{\hfil References Cited\hfil}                   %%
%%%%%%%%%%%%%%%%%%%%%%%%%%%%%%%%%%%%%%%%%%%%%%%%%%%%%%%%%%%%%%%%%%%%%%%%%

%PUT YOUR MACROS HERE
\usepackage{hyperref} %for urls
\usepackage{graphicx} %for includegraphics
\usepackage[table]{xcolor} %for table line color alteration
\usepackage{enumitem}  %fancy enumerated lists
\usepackage{wrapfig} %wrap text around figures
\usepackage[leftcaption]{sidecap} % side captions
\sidecaptionvpos{figure}{c} %position side caption

\usepackage{multirow} %multirow tables
\usepackage{booktabs} %funky tabs in tables
\renewcommand{\thepage}{f. Budget Justification - Page \arabic{page} of 2}

%\includeonly{NSFsumm}

\begin{document}
\setcounter{page}{1}

\required{Budget Justification}
%\begin{center}
%\emph{Maximum of 3 pages}
%\end{center}

\subsection*{Personnel}


\subsection*{Other Personnel}
\subsubsection{Postdoctoral Scholars}
Funds are requested to support a postdoc for 12 months per year, for all three years of the proposal with a base salary of \$42,840. The postdoc would lead the population genetic analysis of GBS and sequence data of introgression and maize-teosinte hybridization.

\subsubsection{Technician}
Funds are requested for the first two years of the grant for 2 calendar months (17\% time) of support for a technician (Assistant Specialist I) to extract DNA and prepare GBS and genomic sequencing libraries, facilitate genotyping/sample prep for collaborating labs, and coordinate the undergraduate laboratory outreach. The base salary for this positions is \$42,144.


\subsection*{Fringe Benefits}
Fringe benefits are applied to personnel salaries using the university approved rates:
\begin{itemize}
\item Postdocs - 17\% in FY 2015, 18\% in 2016, and 19\% in 2017
%\item Graduate students - 1.3\% for all years.
%\item Undergraduate students - \% in FYs 2012, 2013, and 2014
\item Technician - 50.4\%(1/1/2015-6/31/2015), 53.4\%(6/31/2015-6/31/2016), 55.7\%(6/31/2016-6/31/2017), 57.3\%(6/31/2017-12/31/2017)
%\item Part time staff - \% in FYs 2012, 2013, and 2014
\end{itemize}

\subsection*{Equipment}

No equipment funds are requested.

\subsection*{Travel}

Travel for two travelers (the PI or Sr. Personnel and the postdoc) for domestic or international conference travel is budgeted each year at \$3,000. 

\subsection*{Other Direct Costs}

 \paragraph{Materials and Supplies}
In each of the three years of the grant, \$5,000 is requested in materials and supplies for library prep for whole genome sequencing, and DNA extraction and preparation for GBS.  This also includes funds for standard office supplies, computer supplies (extra storage for our cluster, backup drives for lab members), and other miscellaneous expenses. We have also included \$1,000 in supplies to support undergraduate-led research projects.

\paragraph{Whole genome sequencing}
The genomes of each of eight teosinte will be resequenced to a depth of 30X using 13 lanes of paired end 150bp reads on an Illumina HiSeq 3000. Current lane costs are approximately \$2,500 per lane, and library preparations costs are approximately \$75, for a total cost of \$14,146.00 for two \emph{Zea mays} ssp. \emph{huehuetenangensis} and two \emph{Zea mays} ssp. \emph{mexicana} in year 1 and \$18,761 for the larger \emph{Zea luxurians} and \emph{Zea diploperennis} genomes in year 2.

\paragraph{GBS}
Genotyping-by-sequencing will be performed for our introgression analyses admixture population genetic analyses. GBS will be performed at the Institute for Genomic Diversity at Cornell.  Current prices are \$34 per sample to run samples at 96-plex.  We will genotype 288 individuals in year 1 for a cost of \$9,840, and 192 individuals in year 2 for a cost of \$6,560. 

%\paragraph{RNA sequencing}
%In total, RNA sequencing will be performed on 384 individuals (8 inbreds x 2 stages x 2 tissues x 2 environments x 3 replicates + 8 NILs x 2 genotypes x 2 stages x 2 environs x 3 replicates).  Cost to prepare RNA libraries in our lab are approximately \$100 per library, and sequencing costs for single-end 50bp reads at the UCD Genome Center are approximately \$1,000 per lane.  Multiplexing 12 barcodes per lane, this comes out to 32 lanes of sequence and a total cost of \$70,400.

%\paragraph{Field fees}
%Fees for the field experiments in our highland and lowland field sites (Table \ref{tab:locales}) are approximately \$60,000 the first three years of the experiment to allow development of the mapping populations and two replicates of the phenotyping.  These fees include land rental and basic management (planting, watering, weeding, fertilizing), as well as station fees to hire manual labor for phenotyping.  These fees decrease to \$10,000 in the last two years of the proposal as subsequent field experiments including evaluation of NILs and RNA-seq lines, will be considerably smaller. Field fees total \$200,000 across the five years of the grant.

%\paragraph{Graduate Student Tuition}
%Tuition for graduate students is charged to the project in proportion to the amount of effort the graduate student will work on the project. For a graduate student employed on the project for 9 academic months at 50\% FTE, the tuition charge is \$31,546 in FY 2015 to account for out-of-state tuition, \$17,266 in FY 2016 and increasing 5\% each subsequent year.

\paragraph{Publication Costs}
In year two and three \$1,600 is requested for publication fees to an open access journal. 

\subsection*{Total Direct Costs}

Total direct costs for UCD come to \$253,119.  

\subsection*{Indirect Costs}
Indirect costs are calculated on Total Direct Costs using F\&A rates approved by US Department of Health and Human Services. For this project, F\&A rates of 56.5\% were used from July 1, 2015 through June 30, 2016, and 57\% from July 1, 2016 until the end of the project.

\end{document}

