%%%%%%%%% PROJECT DESCRIPTION  -- 15 pages (including Prior NSF Support)

\required{Project Description}
\begin{center}
%\emph{Maximum of 15 pages}
\end{center}
%The Project Description (including Results from Prior NSF Support, which is
%limited to five pages) may not exceed 15 pages. Visual materials, including charts,
%graphs, maps, photographs and other pictorial presentations are included in the
%15-page limitation. PIs be cautioned that the project description must
%be self-contained and that URLs that provide information related to the proposal
%should not be used. \\
%
%All proposals to NSF are reviewed utilizing the two merit review criteria,
%intellectual merit and broader impacts. \\
%
% The Project Description should provide a clear statement of the work 
% to be undertaken and must include: objectives for the period of the proposed 
% work and expected significance; relation to longer-term goals of the PI's 
% project; and relation to the present state of knowledge in the field, 
% to work in progress by the PI under other support and to work in progress 
% elsewhere.

%%%%%%%%%%%%%%%%%%%%%%%%%%%%%%%%%%%%%%%%%%%%%%%%%%%%%%%%%%%%%%%%%%%%%%
%INTRO
%%%%%%%%%%%%%%%%%%%%%%%%%%%%%%%%%%%%%%%%%%%%%%%%%%%%%%%%%%%%%%%%%%%%%%
\section*{Introduction}

	While the potential roles of hybridization and introgression as agents of evolution have long been appreciated (Anderson 1948; Anderson \& Stebbins 1954; Stebbins 1959), only recently have technological innovations begun to allow their characterization on a genome-wide scale. Multiple genome-wide studies have now reported substantial inter-taxon introgression in both plant (Hufford et al. 2013; Renaut et al. 2013) and animal (Dasmahapatra et al. 2012; Staubach et al. 2012) species. In the investigation proposed here, we will build upon our previous work in maize (\emph{Zea mays} ssp. \emph{mays}) and its wild relatives (\emph{Zea} spp.; collectively referred to as teosinte) in order to characterize the evolutionary role of hybridization and introgression in the genus \emph{Zea}. This study system is ideally suited for such research due to its well-characterized history, large natural populations, and the availability of exceptional genomic resources (Hufford et al. 2012). 

%%%%%%%%%%%%%%%%%%%%%%%%%%%%%%%%%%%%%%%%%%%%%%%%%%%%%%%%%%%%%%%%%%%%%%
%AIMS
%%%%%%%%%%%%%%%%%%%%%%%%%%%%%%%%%%%%%%%%%%%%%%%%%%%%%%%%%%%%%%%%%%%%%%
\section*{Objectives}
	
We will leverage the many positive attributes of the \emph{Zea} study system to address two primary objectives:

\subsection*{Objective I) Assess the evolutionary significance of hybridization and introgression in locally-adapted, parapatric wild Zea taxa}

	Zea mays ssp. parviglumis (the wild progenitor of maize; hereafter, parviglumis) and Zea mays ssp. mexicana (hereafter, mexicana) diverged approximately 60,000 BP (Ross-Ibarra et al. 2009) and have parapatric distributions: while parviglumis occurs in the warm lowlands of southwest Mexico, mexicana is found in the cool highlands of the Mexican Central Plateau. Narrow regions of substantial admixture between these wild subspecies have been discovered at middle elevations (Fukunaga et al. 2005; Pyh�j�rvi et al. 2013). Through targeted collections, generation of high-density genotyping data, population genomic analyses, and common garden experiments, we will address the following research questions and work toward establishing this system as a model for the study of hybrid zones:

\subsubsection*{\emph{Research Question IA) What fraction of the genome is porous to gene flow in hybrid zones?}}
\subsubsection*{\emph{Research Question IB) How do parviglumis, mexicana, and hybrid fitness vary across the hybrid zone?}}
\subsubsection*{\emph{Research Question IC) Are allele frequency clines at loci associated with fitness traits in mexicana and parviglumis steeper than those at non-associated loci?}}

\subsection*{Objective II) Determine the extent to which hybridization and introgression have altered the Zea genus during the post-domestication spread of maize}

	Maize was domesticated in southwest Mexico from parviglumis ~10,000 BP (Matsuoka et al. 2002) and quickly spread thoughout North America, bringing this crop into sympatry with new species of teosinte (Vigouroux et al. 2008). Through dense genotyping of range-wide samples of maize and teosinte, we will assess two questions regarding the importance of introgression during the spread of maize:
	
\subsubsection*{\emph{Research Question IIA) Did maize serve as a bridge for gene flow between previously isolated Zea taxa?}}
\subsubsection*{\emph{Research Question IIB) Was the spread of maize facilitated by gene flow from locally-adapted wild Zea?}}


%%%%%%%%%%%%%%%%%%%%%%%%%%%%%%%%%%%%%%%%%%%%%%%%%%%%%%%%%%%%%%%%%%%%%%
%RATIONALE AND SIGNIFICANCE
%%%%%%%%%%%%%%%%%%%%%%%%%%%%%%%%%%%%%%%%%%%%%%%%%%%%%%%%%%%%%%%%%%%%%%
\section*{Rationale and Significance}
Pioneers in evolutionary biology including G. Ledyard Stebbins and Edgar Anderson recognized the important role hybridization and introgression could play in adaptation and speciation (Anderson 1948; Anderson \& Stebbins 1954). These evolutionary forces were thought to be particularly influential when environmental conditions for a species were marginal, variable, or new (Stebbins 1959). More recently, theoretical and empirical investigations of hybridization have focused on hybrid zones, defined as regions where distinct taxa co-occur and mate, resulting in progeny of mixed ancestry (Harrison 1993). While hybrid zone theory has progressed and many compelling empirical examples have been identified based on hybrid morphology and limited genetic data (Delmore et al. 2013; Galindo et al. 2013; Parchman et al. 2013; Smith et al. 2013b), many outstanding questions remain. For example, additional study is needed to determine whether hybrid zones are primarily maintained as tension zones in which hybrids are selected against or as ecotones where hybrids have an advantage under certain environmental conditions (Kruuk et al. 1999; Rasmussen et al. 2012; Smith et al. 2013a). Additionally, genome-wide analysis of the fraction of the genome that is porous to gene flow in hybrid zones is rare and will likely offer considerable insight.
	 
Both wild and domesticated Zea offer exciting opportunities to study hybridization. Parviglumis and mexicana are distributed across a steep altitudinal gradient and differ for putatively adaptive phenotypes such as the presence of macrohairs and stem pigmentation found in mexicana. Analysis of ~100 microsatellite markers genotyped in a range-wide sample of parviglumis and mexicana identified clear regions of elevated admixture in two geographically-distinct, mid-elevation regions: eastern Jalisco state and the eastern Balsas River Basin (Fukunaga et al. 2005). Our genome-wide analysis (~40,000 SNPs) of a hybrid population in the eastern Balsas revealed extensive admixture (Pyh�j�rvi et al. 2013), and the relatively short length of shared haplotypes suggested continual gene flow between parviglumis and mexicana over a substantial period of time (Pyh�j�rvi et al. 2013). Furthermore, a reanalysis of a data set of 1000 SNPs from samples across the range of both taxa (Fang et al. 2012) reveals substantial admixture in multiple populations at middle elevation (Figure 1).
	 
Because of its recent domestication and rapid spread, hybridization between maize and other Zea must be quite young from an evolutionary perspective. Our recent work has provided convincing evidence of introgression from mexicana into maize at QTL for phenotypes (e.g., pigment and macrohairs) that distinguish highland mexicana from lowland parviglumis (Hufford et al. 2013). Our interpretation of this result is that maize received adaptive introgression from mexicana that allowed the crop to spread into the highlands of Mexico. Subsequent to its diffusion across the Mexican highlands, maize spread into sympatry with two additional teosinte taxa in Guatemala, Zea mays ssp. huehuetenangensis (hereafter, huehuetenangensis) and Zea luxurians (hereafter, luxurians). Hybrids between maize and the Guatemalan teosinte taxa have been observed in the field (Wilkes 1977) and evidence has been found of mexicana haplotypes segregating in luxurians (Ross-Ibarra et al. 2009).  Mexicana and luxurians are entirely allopatric in their distributions, which suggests maize has served as a bridge for gene flow between these two taxa. 

%%%%%%%%%%%%%%%%%%%%%%%%%%%%%%%%%%%%%%%%%%%%%%%%%%%%%%%%%%%%%%%%%%%%%%
%PRELIMINARY RESULTS
%%%%%%%%%%%%%%%%%%%%%%%%%%%%%%%%%%%%%%%%%%%%%%%%%%%%%%%%%%%%%%%%%%%%%%
\section*{Preliminary Results}
%%%%%%%%%%%%%%%%%%%%%%%%%%%%%%%%%%%%%%%%%%%%%%%%%%%%%%%%%%%%%%%%%%%%%%
%SPECIFIC OBJECTIVES
%%%%%%%%%%%%%%%%%%%%%%%%%%%%%%%%%%%%%%%%%%%%%%%%%%%%%%%%%%%%%%%%%%%%%%
\section*{Specific Objectives}
%%%%%%%%%%%%%%%%%%%%%%%%%%%%%%%%%%%%%%%%%%%%%%%%%%%%%%%%%%%%%%%%%%%%%%
%AIM 1
%%%%%%%%%%%%%%%%%%%%%%%%%%%%%%%%%%%%%%%%%%%%%%%%%%%%%%%%%%%%%%%%%%%%%%
%%%%%%%%%%%%%%%%%%%%%%%%%%%%%%%%%%%%%%%%%%%%%%%%%%%%%%%%%%%%%%%%%%%%%%
%AIM 2
%%%%%%%%%%%%%%%%%%%%%%%%%%%%%%%%%%%%%%%%%%%%%%%%%%%%%%%%%%%%%%%%%%%%%%
%%%%%%%%%%%%%%%%%%%%%%%%%%%%%%%%%%%%%%%%%%%%%%%%%%%%%%%%%%%%%%%%%%%%%%
%AIM 3
%%%%%%%%%%%%%%%%%%%%%%%%%%%%%%%%%%%%%%%%%%%%%%%%%%%%%%%%%%%%%%%%%%%%%%
\required{Broader Impacts}

	Our efforts to broaden the impact of the research proposed here will begin within our groups through our commitment to effectively mentor volunteer undergraduate interns as well as graduate students and/or postdoctoral scholars funded by the project. Students and postdocs will receive one-on-one training from the investigators and senior personnel on laboratory, computational, and field research methods.  Mentees will also be encouraged and funded to present their work at scientific conferences.  Our groups have an excellent mentoring track record with four undergraduate students in the last five years publishing their work in scholarly journals and multiple underrepresented minorities participating in our research.
In addition to the student and postdoc mentoring that will occur within our groups, as part of our broader impact activities one of our graduate students will participate in Iowa State University's GK12 Fellowship program. The selected graduate student will spend one full day each week in a science middle or high school classroom for the entire academic year in the Des Moines Public School District. This is the largest and most diverse school district in Iowa. The graduate student will introduce the K12 students to the scientific process through inquiry-based activities, relate the students� science curriculum to real world examples, work with students on their science fair projects, and serve as a role model in a STEM profession. Furthermore, the graduate student will introduce students to his/her research project on hybridization and introgression in Zea, a topic that is particularly well suited for teaching evolution in Iowa given the important role that maize plays in the Iowan economy. In introducing his/her dissertation research, the graduate student will engage Des Moines students in how research is conducted and provide STEM content professional training to his/her partner teacher. 
	Finally, we will establish a student exchange program between the Eguiarte Laboratory at UNAM in Mexico and the Hufford and Ross-Ibarra Laboratories in the United States. The Ross-Ibarra Laboratory has run an NSF-supported, US-Mexico exchange program for the last three years with two students earning authorship on forthcoming papers and we will build upon the success of this program.  A student from the Eguiarte group will spend 2-3 months in either the Hufford or Ross-Ibarra Laboratory learning the GBS methodology and/or honing his/her skills in population genomic analysis, whereas a student from the Hufford and/or Ross-Ibarra Laboratories will travel to Mexico to participate in sample collection trips and to obtain expertise in common garden field experiments. This exchange will build capacity in all groups involved and will provide a valuable international research experience for a graduate student supported by the grant.

\required{Results From Prior NSF Support}
% 5 pages or fewer of the 15 pages for entire description document.
% include results from NSF grants received in the past 5 years.
% If supported by more than one grant, choose the most relevant one.

% For each grant, include: 
%	(a) NSF award number, amount, dates of support 
%	(b) The title of this project
%	(c) Publications resulting from this research
%	(d) Summary of the results of the completed work
%	(e) A brief description of data samples available and other research products not described 	      elsewhere
%	(f) For renewed support, a description of the relationship between the completed and 			      proposed work

% Due to space limitations, it is often advisable to use citations rather
% than putting the titles of the publications in the body 
% of this section

\subsection*{Ross-Ibarra, Flint-Garcia: \#1238014: Biology of Rare Alleles in Maize and Its Wild Relatives}
\$13,311,185 (\$2,368,767 to Ross-Ibarra and \$1,206,211 to Flint-Garcia), 05/15/13-04/30/18. PI Edward Buckler, co-PIs J. Doebley, J. Holland, S. Flint-Garcia, Q. Sun, P. Bradbury, S. Mitchell, J. Ross-Ibarra
\par\noindent{\bf Intellectual merit} In the first year we have developed accurate imputation approaches, found evidence for the importance of deleterious variants and non-genic polymorphisms in heterosis and GWAS, documented differences in recombination among the parents of the NAM population, and found population genetic evidence suggesting the importance of demography and purifying selection across the genome.  The grant has produced 18 total publications in its first year (only publications involving PIs Flint-Garcia and Ross-Ibarra are shown below). 
\par\noindent{\bf Broader impacts}  In the first year this project has included 10 postdoctoral and 12 graduate trainees. The GBS workshop and traveling maize exhibit continue to be popular and successful. A new version of the teacher-friendly guide to maize evolution has been revised and published online. 
\par\noindent{\bf Publications} \citet{peiffer2013genetic, Romay2013, wills2013many, Mezmouk2014, Peiffer2014, sood2014mining}

\subsection*{Ross-Ibarra: \#0922703: Functional Genomics of Maize Centromeres}
\$5,008,031 (\$754,409 to Ross-Ibarra). 09/01/09-08/31/14. PI Kelly Dawe, co-PIs J. Birchler, J. Jiang, G. Presting, J. Birchler, J. Ross-Ibarra
\par\noindent{\bf Intellectual merit} Centromeres are regions of the genome that organize and regulate chromosome movement, yet the biology of centromeres remains poorly understood. Co-PI Ross-Ibarra's group has focused in particular on the evolutionary genetics of centromeres. This work has demonstrated the remarkable evolutionary lability of centromere tandem repeats, but has shown that there is little evidence in maize for coevolution between centromere sequence and kinetochore proteins. Ongoing work from the Ross-Ibarra lab seeks to characterize kinetochore proteins, assess the phylogenetic evidence for longer-term coevolution, and understand patterns of centromere and genome size variation in natural populations.
\par\noindent{\bf Broader impacts}  Co-PI Ross-Ibarra has established an international student exchange program as part of this grant. Data and result of this project have been disseminated via publications and presentations as well as deposited in the maize genetics community database \url{www.maizegdb.org}. Former trainees on the grant include Dr. Matthew Hufford (Co-PI on the current grant). 
\par\noindent{\bf Publications} \citet{Shi2010a, Chia2012a, Fang2012a, Hufford2012, Hufford2012b, Hufford2013, Melters2013a, Kanizay2013, Pyhajarvi2013}

