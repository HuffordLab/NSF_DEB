%%%%%%%%% PROJECT DESCRIPTION  -- 15 pages (including Prior NSF Support)

\required{Project Description}
\begin{center}
%\emph{Maximum of 15 pages}
\end{center}
%The Project Description (including Results from Prior NSF Support, which is
%limited to five pages) may not exceed 15 pages. Visual materials, including charts,
%graphs, maps, photographs and other pictorial presentations are included in the
%15-page limitation. PIs be cautioned that the project description must
%be self-contained and that URLs that provide information related to the proposal
%should not be used. \\
%
%All proposals to NSF are reviewed utilizing the two merit review criteria,
%intellectual merit and broader impacts. \\
%
% The Project Description should provide a clear statement of the work 
% to be undertaken and must include: objectives for the period of the proposed 
% work and expected significance; relation to longer-term goals of the PI's 
% project; and relation to the present state of knowledge in the field, 
% to work in progress by the PI under other support and to work in progress 
% elsewhere.

%%%%%%%%%%%%%%%%%%%%%%%%%%%%%%%%%%%%%%%%%%%%%%%%%%%%%%%%%%%%%%%%%%%%%%
%INTRO
%%%%%%%%%%%%%%%%%%%%%%%%%%%%%%%%%%%%%%%%%%%%%%%%%%%%%%%%%%%%%%%%%%%%%%
\section*{Introduction}

While the potential roles of hybridization and introgression as agents of evolution have long been appreciated \citep{Anderson1948, Anderson1954, Stebbins1959}, only recently have technological innovations allowed for characterization of these processes on a genome-wide scale. 
Multiple studies have now reported substantial inter-taxon introgression in both plant \citep{Hufford2013, renaut2013} and animal \citep{consortiumbutterfly2012, staubach2012, huerta2014} species, and introgression has been found across considerable portions of genomes and specifically at loci thought to underlie adaptation. 

In the investigation proposed here, we will build upon our previous work in maize (\emph{Zea mays} ssp. \emph{mays}) and its wild relatives the teosintes (\emph{Zea} spp.) to characterize the evolutionary role of hybridization and introgression. 
As a study system, the genus \emph{Zea} is ideally suited for such research due to its large natural populations and the availability of exceptional genomic resources \citep{Hufford2012}.  Within \emph{Zea} we will be able to assess the genome-wide effects of hybridization and introgression at two different timescales.  
First, analysis of hybridization and divergence between the subspecies \emph{Zea mays} ssp. \emph{parviglumis} and \emph{Zea mays} ssp. \emph{mexicana} will generate basic knowledge of the process of incipient speciation and the porous nature of genomes of diverging species on an evolutionary timescale ($\sim60,000$ generations).
Second, an evaluation of introgression in sympatric maize and teosinte populations on the ecological timescale ($\sim3,000-9,000$ generations) during which domesticated maize colonized the Americas will inform our understanding of the role of introgression in adaptation to new or rapidly changing environments.   

%%%%%%%%%%%%%%%%%%%%%%%%%%%%%%%%%%%%%%%%%%%%%%%%%%%%%%%%%%%%%%%%%%%%%%
%OBJECTIVES
%%%%%%%%%%%%%%%%%%%%%%%%%%%%%%%%%%%%%%%%%%%%%%%%%%%%%%%%%%%%%%%%%%%%%%
\section*{Objectives}
	
%We will leverage the resources of the \emph{Zea} study system to address two primary objectives:

\subsection*{Objective I: Assess the evolutionary and genomic effects of hybridization in locally-adapted, parapatric wild teosinte}
\emph{Zea mays} ssp. \emph{parviglumis} (the wild progenitor of maize; hereafter, \emph{parviglumis}) and \emph{Zea mays} ssp. \emph{mexicana} (hereafter, \emph{mexicana}) diverged approximately 60,000 BP \citep{Ross-Ibarra2009a} and have parapatric distributions: while \emph{parviglumis} occurs in the warm lowlands of southwest Mexico, \emph{mexicana} is found in the cool highlands of the Central Plateau. Narrow regions of admixture between these wild subspecies have been discovered at middle elevations \citep{Fukunaga2005, Pyhajarvi2013}. Through targeted collections, generation of high-density genotyping data, population genomic analyses, and common garden experiments, we will address the following research questions:
\begin{enumerate}
\item \emph{What fraction of the genome is porous to gene flow in hybrid zones?}

\item \emph{How do the fitness of parental and hybrid populations vary across the hybrid zone?}

\item \emph{Is there evidence of selection on putatively adaptive traits across hybrid zones?}
\end{enumerate}
\vspace{0.5cm}

\subsection*{Objective II: Determine the extent to which hybridization and introgression have altered the \emph{Zea} genus during the post domestication spread of maize}
Maize was domesticated in southwest Mexico from \emph{parviglumis} $\sim$9,000 BP \citep{Matsuoka2002} and quickly spread throughout the Americas, bringing this crop into sympatry with new species of teosinte \citep{Vigouroux2008a}. Through a combination of dense genotyping of range-wide samples of maize and teosinte and targeted, full-genome sequencing, we will assess three questions regarding the importance of introgression during the spread of maize:
\begin{enumerate}
\item \emph{Was the spread of maize facilitated by gene flow from locally-adapted wild \emph{Zea}?}
\item \emph{What is the geographic scale of adaptive introgression?}
\item \emph{Did maize serve as a bridge for gene flow between previously isolated \emph{Zea} taxa?}
\end{enumerate}

%%%%%%%%%%%%%%%%%%%%%%%%%%%%%%%%%%%%%%%%%%%%%%%%%%%%%%%%%%%%%%%%%%%%%%
%RATIONALE AND SIGNIFICANCE
%%%%%%%%%%%%%%%%%%%%%%%%%%%%%%%%%%%%%%%%%%%%%%%%%%%%%%%%%%%%%%%%%%%%%%
\section*{Rationale and Significance}
Pioneers in evolutionary biology including G. Ledyard Stebbins and Edgar Anderson recognized the important role hybridization and introgression could play in adaptation and speciation \citep{Anderson1948, Anderson1954}. These evolutionary forces were thought to be particularly influential when environmental conditions encountered by a species were marginal, variable, or new \citep{Stebbins1959}. More recently, theoretical and empirical investigations of hybridization have focused on hybrid zones, defined as regions where distinct taxa co-occur and mate, resulting in progeny of mixed ancestry \citep{HarrisonHybridZone}. 

While hybrid zone theory has progressed and many compelling empirical examples have been identified based on hybrid morphology and genetic marker data \citep{Delmore2013, Galindo2013, Parchman2013, Smith2013a}, several outstanding questions remain. 
Additional study is needed to determine whether hybrid zones are primarily maintained as tension zones in which hybrids are selected against or as ecotones where hybrids have an advantage under certain environmental conditions \citep{Kruuk1999, Rasmussen2012, Smith2013b}. 
Moreover, genome-wide analysis of the fraction of the genome that is porous to gene flow in hybrid zones is rare and will likely offer considerable insight.  
For example, recent genomic studies of introgression suggest that rates of gene flow vary substantially across loci, likely as a function of selection for or against introgressed alleles \citep{Hufford2013, Poelstra2014}.  
In addition, chromosomal rearrangements (\emph{e.g.}, inversions and translocations) likely play an important role in adaptation and structuring introgression along the genome and may restrict gene flow in hybridizing species \citep{Barb2014, guerrero2014}.
	 
Both wild and domesticated \emph{Zea} offer exciting opportunities to study hybridization. 
The subspecies \emph{parviglumis} and \emph{mexicana} are distributed across a steep altitudinal gradient and differ for traits that are thought to be adaptive in the highlands such as the presence of macrohairs and stem pigmentation.
A recent ecological niche study has found that distributions of these subspecies are quite unique and stable over many thousands of years \citep{hufford2012inferences}.
However, analysis of microsatellite markers genotyped in a range-wide sample has identified elevated admixture between the subspecies in two geographically distinct mid-elevation regions of Mexico between the distributions of both parental species, suggesting the presence of multiple hybrid zones \citep{Fukunaga2005}.  
Our recent genome-wide analysis of a population in one of these hybrid zones revealed extensive subspecies admixture across all individuals sampled and relatively short shared haplotypes with individuals from other populations of either \emph{parviglumis} or \emph{mexicana} \citep{Pyhajarvi2013}.
This suggests continual gene flow between \emph{parviglumis} and \emph{mexicana} in this hybrid zone over a substantial period of time \citep{Pyhajarvi2013}.  
Longer shared haplotypes were found in the hybrid population in chromosomal regions identified as potential inversions \citep{Pyhajarvi2013}.  
These regions may be particularly resistant to gene flow between subspecies, but may also prove adaptive for individuals in some parts of the hybrid zone.
Findings from this single hybrid population suggest that gene flow between taxa is ongoing yet varies in extent across the genome.
However, very little is known regarding genome-wide patterns of hybridization and subsequent introgression in other populations in this region or in other hybrid zones.
Hybrid populations of \emph{mexicana} and \emph{parviglumis} are distributed across elevation gradients in markedly different regions of Mexico and are found at varying distances from each subspecies.
By expanding our study of hybridization and introgression across these taxa we can compare evidence supporting ecotone and tension zone hypotheses and, should hybridization prove adaptive, assess how the genomic landscape of hybridization is shaped by natural selection.

In addition to the ongoing hybridization between \emph{parviglumis} and \emph{mexicana} occurring since divergence $\sim$60,000 BP, gene flow between domesticated maize and various taxa of the genus \emph{Zea} has been detected based on both hybrid morphologies observed in the field \citep{wilkes1967teosinte, Wilkes1977} and genetic data \citep{Fukunaga2005,Ross-Ibarra2009a}. 
Maize domestication from \emph{parviglumis} occurred recently on an evolutionary timescale \citep[$\sim$9,000BP][]{Matsuoka2002}) and was followed by rapid spread of the crop across the Americas over the following millennia \citep{Piperno2001,Grobman2012}. 
During this diffusion, maize was brought into sympatry with new wild relatives that were likely allopatric to the progenitor of maize (\emph{i.e., parviglumis}) for long periods prior to domestication \citep{hufford2012inferences}. 
Our recent work has provided evidence of introgression from \emph{mexicana} into maize during its earliest colonization of the highlands of the Mexican Central Plateau.  
We found consistent introgression into several highland maize populations at QTL for phenotypes (\emph{e.g.}, pigment and macrohairs) that distinguish highland \emph{mexicana} from lowland \emph{parviglumis} teosinte, and showed differences in these phenotypes as well as growth rate under cold conditions between maize plants with and without \emph{mexicana} introgression \citep{Hufford2013}.
Our interpretation of these results is that maize received adaptive introgression from \emph{mexicana} that allowed the crop to spread into the highlands of Mexico. 

Subsequent to its diffusion into the Mexican highlands, maize spread into sympatry with additional teosinte taxa in Guatemala including \emph{Zea luxurians} (hereafter, \emph{luxurians}) and \emph{Zea mays} ssp. \emph{huehuetenangensis} (hereafter, \emph{huehuetenangensis}), each adapted to environmental conditions very different from those of \emph{parviglumis}. Although maize is known to hybridize with both taxa, the extent and adaptive significance of gene flow between these teosintes and maize is unknown.  
Based on analysis of a small number of resequenced loci, \emph{mexicana} haplotypes appear to be segregating in \emph{luxurians} \citep{Ross-Ibarra2009a}.  Since \emph{mexicana} and \emph{luxurians} are entirely allopatric in their distributions, this suggests maize may have served as a bridge for gene flow between these two taxa.  
Further work will be necessary to explore this possibility and to assess if maize has, more generally, altered the genomes of \emph{Zea} species through gene flow during its spread across the Americas (see \textbf{Objective II}). 

%\jri{this section also reads long. compare to our section in PGRP. i think this is more background info than is needed and less why this is important than is needed.} \mbh{tried to address this a bit and incorporate your other comments (good comments, BTW).  If more cutting down is needed, perhaps we can come back to this once we've finished other edits and the research plan is set?}

%%%%%%%%%%%%%%%%%%%%%%%%%%%%%%%%%%%%%%%%%%%%%%%%%%%%%%%%%%%%%%%%%%%%%%
%PRELIMINARY RESULTS
%%%%%%%%%%%%%%%%%%%%%%%%%%%%%%%%%%%%%%%%%%%%%%%%%%%%%%%%%%%%%%%%%%%%%%
\section*{Preliminary Results}

Our previous publications suggest \emph{Zea} is a promising model system for exploring the evolutionary role of hybridization and introgression (\emph{e.g.},  \citealt{Ross-Ibarra2009a, vanheerwaarden2011a, Hufford2013, Pyhajarvi2013}).  To further refine our research questions and provide preliminary results for this proposal we have reanalyzed published data  \citep{Fang2012} of 983 SNPs genotyped across a panel of $>2,000$ samples including all subspecies and species of teosinte and an Americas-wide sample of maize landraces (\emph{i.e.}, traditional open-pollinated varieties).  While the low density of markers in these data precludes genome-wide inferences and haplotype-based analyses, the comprehensive taxon sampling makes this an ideal resource for guiding future research.
\jri{that last sentence is awesome. well said.}

\subsection*{Evidence for hybrid zones between \emph{parviglumis} and \emph{mexicana}}

Using the \citet{Fang2012} data, we calculated the probability of each sample's assignment to \emph{parviglumis} and \emph{mexicana} groups using STRUCTURE \citep{Pritchard2000}.  We find that individuals from several mid-elevation populations show appreciable assignment to both \emph{parviglumis} and \emph{mexicana} groups (Figure \ref{fig:structure}) and likely represent hybrid populations.  

\begin{figure}[t]
  \centering
   \includegraphics[width=0.95\textwidth]{structure.pdf}
    \caption{Assignment of \emph{parviglumis} and \emph{mexicana} individuals to K=2 groups using the Bayesian assignment algorithm of STRUCTURE \citep{Pritchard2000}.  Individuals are sorted by increasing altitude as indicated by the plot above the bar chart. Individuals from mid-elevation, hybrid zone populations are underscored in green.} 
\label{fig:structure}
\end{figure}

Several admixed populations cluster in two geographically distinct regions of Mexico: the eastern Balsas River Basin and eastern Jalisco state.  These locations fall at intermediate locations between the main distributions of \emph{parviglumis} and \emph{mexicana} (Panel A, Figure \ref{fig:pies}).  Hybrid populations from eastern Jalisco state are found at higher elevation (mean 1632m) than those in the eastern Balsas (mean 1531m) and also show a higher proportion of membership in the highland teosinte \emph{mexicana} (Panels B and C, Figure \ref{fig:pies}).  These findings suggest that hybrid populations from distinct environments may vary in the proportion of ancestry from these two subspecies in a manner that is adaptive. % Estimates of pairwise population differentiation also suggest that hybrid populations in the Balsas and Jalisco are distinct in that Jalisco populations are less differentiated from \emph{mexicana} than hybrid populations in the Balsas (Table \ref{tab:Fst}).  Not surprisingly, populations in both hybrid zones are less differentiated from \emph{mexicana} and \emph{parviglumis} than these subspecies are from each other. 
\jri{not sure the Fst helps, we can easily ditch to save space. i commented out in text.}

\begin{figure}[h]
  \centering
   \includegraphics[width=0.7\textwidth]{map.pdf}
    \caption{A) Location of two putative hybrid zones of \emph{mexicana} and \emph{parviglumis}.  Hybrid populations are represented as pie charts with proportion assigned to \emph{mexicana}, \emph{parviglumis}, and maize groups. Zoomed-in views of the Eastern Jalisco (B) and Eastern Balsas (C) hybrid populations.} 
\label{fig:pies}
\end{figure}

%\begin{table}[h]
%\rowcolors{2}{gray!25}{white}
%\begin{center}
%\caption{Pairwise $F_{ST}$ between teosinte and hybrid populations} \label{tab:Fst}
%\begin{tabular}{lcccc}\\\toprule
%{\bf Taxon}&{\bf \emph{parviglumis}}&{\bf \emph{mexicana}}&{\bf Jalisco Hybrids}&{\bf Balsas Hybrids}\\\midrule
%{\bf \emph{parviglumis}}&---&---&---&---\\
%{\bf \emph{mexicana}}&0.107&---&---&---\\
%{\bf Jalisco Hybrids}&0.059&0.064&---&---\\
%{\bf Balsas Hybrids}&0.057&0.074&0.034&---\\\bottomrule
%\end{tabular}
%\end{center}
%\end{table} 

\subsection*{Evidence for introgression across \emph{Zea}}

TreeMix Results\\



\begin{figure}
  \centering
   \includegraphics[width=0.7\textwidth]{fakemap}
    \caption{A) Inv4m hapmix from Matt's paper B) map of Inv4m SNP from David} 
\label{fig:hufmap}
\end{figure}


%%%%%%%%%%%%%%%%%%%%%%%%%%%%%%%%%%%%%%%%%%%%%%%%%%%%%%%%%%%%%%%%%%%%%%
%SPECIFIC OBJECTIVES
%%%%%%%%%%%%%%%%%%%%%%%%%%%%%%%%%%%%%%%%%%%%%%%%%%%%%%%%%%%%%%%%%%%%%%
\section*{Research Plan}

\subsection*{Objective I: Hybrid Zones}

\subsubsection*{A) What fraction of the genome is porous to gene flow in hybrid zones?}

Recent empirical investigations have suggested that functional architecture of genomes can lead to islands of speciation (\emph{i.e.}, regions of high differentiation) in hybridizing species \citep{renaut2013}, but that these regions are not always the same across hybrid populations \citep{Parchman2013}.  Currently, very few studies have dissected the genome-wide architecture of hybridization and introgression in replicate hybrid zones, and very little is known about the consistency of genome porosity to gene flow. Genome-wide studies in teosinte are feasible at very high marker density \citep{Hufford2012b, Hufford2013, Pyhajarvi2013} and are also informed by the genomic resources of maize \citep{Hufford2012}, often providing detailed functional annotation for loci of interest, a rarity in other natural systems.  We will assess the genomic architecture of hybridization in two putative hybrid zones of \emph{mexicana} and \emph{parviglumis} through careful collections and assembly of a study panel, generation of genome-wide marker and sequence data, and application of recently-developed population genomic analyses appropriate to this question.

\paragraph{\emph{Panel Construction and Sample Collection:}}
From previous collections, we have access to an extensive sampling of \emph{parviglumis}, \emph{mexicana}, and maize from throughout their respective ranges.  Moreover, Senior Personnel Luis Eguiarte and collaborator Salvador Montes-Hernandez (see attached letter of commitment) have recently collected altitudinal transects of \emph{parviglumis} and \emph{mexicana} that extend through both hybrid zones targeted by our project \citep{Diez2013} and are  familiar with populations in this region.  Our current collections will likely be insufficient for both the genotyping and common garden activities we propose  and we have therefore budgeted for a collection trip during the first year of the project.  We will collect from 15 sampling sites \jri{what is a sampling site? how much seed? how many dudes will you GBS? why not just say how many individuals you will collect?} in each of 16 populations.  Four populations will be sampled from both the eastern Jalisco and eastern Balsas hybrid zones.  To the extent possible, we will select hybrid populations in a stratified manner across the elevation gradient found in these regions. Four populations each will also be collected from non-admixed \emph{parviglumis} and \emph{mexicana}, with two populations of each taxon collected from regions proximate to both hybrid zones.  We have already obtained the requisite collection permits as well as permits for importing samples into the United States for genetic analysis. Following collection, samples will be divided with half being sent to the lab of Dr. Eguiarte at the Universidad Nacional Aut\'{o}noma de M\'{e}xico (UNAM) for common garden experiments (see Research Question IB) and half being sent to Iowa State University for DNA isolations and subsequent genotyping and full-genome sequencing. \jri{this is weird a bit. its not hard to collect samples from plants in the field. that would let you do gwas in addition to e.g. driftsel}

\paragraph{\emph{Sample Genotyping and Sequencing:}}
DNA for genotyping will be isolated using a modified CTAB procedure \citep{Saghai-Maroof1984}.  For sample genotyping ($n=240$) we will utilize a reduced representation approach to next-generation sequencing called Genotyping By Sequencing (GBS; \citealt{Elshire2011}). To date, this method has been implemented to genotype tens of thousands of maize samples and a bioinformatics pipeline (TASSLE-GBS) has been constructed that allows for genotyping $\sim$1,000,000 SNPs in maize \citep{Glaubitz2014} using standard GBS data. 
We have already successfully applied GBS to heterozygous maize and teosinte and find that, even after filtering for missing data, GBS provides many more markers with minimal ascertainment bias at a fraction of the cost of other available technologies. \jri{we could in theory put sho's paper on fighsare or someplace and get a doi and cite it for evidence we know how to use GBS. worth it?}

In addition to GBS data, we will generate full-genome sequence through the Iowa State University DNA Facility for a single hybrid individual from each hybrid zone.  We will generate two lanes of Illumina HiSeq 150bp, paired-end data per individual.  
We have previous experience dealing with whole genome shotgun data \citep{Gore2009,Chia2012a,Hufford2012b}, and have recently developed and implemented an open-source pipeline for read-mapping and SNP calling (\url{https://github.com/RILAB/paap}) using the existing maize B73 reference genome.

\paragraph{\emph{Population Genomic Analyses:}}
We will assess the genome-wide patterns of porosity to gene flow in hybrid zones using both standard and newly developed population genomic analytical methods that are appropriate to this question.  Standard measures of differentiation and introgression will include $F_{ST}$, the proportion of shared and fixed variants in hybrid versus reference allopatric populations, Reich's $f$ statistics \citep{reich2009}. \jri{i ditched }genome-wide LD as that doesn't really do much by itself} We will also implement haplotype-based methods for detecting hybridization and introgression \citep[\emph{e.g.},][]{price2009, lawson2012} that will effectively allow us to model chromosomes from hybrid populations as mosaics of reference allopatric populations of \emph{parviglumis} and \emph{mexicana}.  We will assess excess of \emph{parviglumis} or \emph{mexicana} ancestry on a site-by-site basis across hybrid genomes at the population level and will determine whether patterns are conserved across populations within hybrid zones and between hybrid zones.  Chromosomal regions showing an excess of ancestry from one taxon in hybrid populations will be inspected for evidence of selection using a combination of site-frequency-, linkage-disequilibrium-, and population-differentiation-based methods \citep[reviewed in][]{Vitti2013}. Chromosomal regions showing strong evidence of selection across individuals within a hybrid zone based on analysis of GBS data will be further dissected using high-density, full-genome data generated for a single individual per hybrid zone.  
Whole genome sequence will allow definition of the exact haplotype(s) that have introgressed, allowing estimation of the age of the introgression and potentially identifying candidate causal polymorphisms.  

\subsubsection*{B) How do the fitness of parental and hybrid populations vary across the hybrid zone?} \jri{reading this now i kind of think switching B and A here and leading with common garden would be good.  everyone is sequencing everything these days, doing the hard common garden work is one thing that will set us apart}

In order to assess fitness and variation at putatively adaptive traits across both non-admixed and  hybrid populations we will conduct common garden experiments in Mexico at three altitudes: 1) Below a hybrid zone in habitat occupied by non-admixed \emph{parviglumis}; 2) Within hybrid zone habitat; and 3) Above a hybrid zone in habitat occupied by non-admixed \emph{mexicana}. 
Common garden experiments will be replicated over years two and three of our proposed project.  

Ddiscussions with collaborators in Mexico (Ruairidh Sawers and Salvador Montes-Hernandez; see attached letters of commitment) raised concerns about the safety of students at field sites in the state of Guerrero (the location of the eastern Balsas hybrid zone), and we thus propose a single transect of three gardens in the eastern Jalisco hybrid zone. 
In our initial discussions with Drs. Sawers and Montes-Hernandez we have identified potential high- and low-elevation sites near Celaya and Bucerias, Mexico respectively.  
We will explore options for our third garden in the hybrid zone during our collections in the first year of the project. 
Each garden will consist of three complete blocks including a randomization of three plants from each of 15 sampling sites in the 16 populations described in Objective IA (3 blocks x 3 plants x 15 sites x 16 populations = 2,160 plants per site).  Our experiments will gauge relative fitness between \emph{parviglumis}, \emph{mexicana}, and hybrids at each of these sites. 
We will measure fitness-related phenotypes (percent germination, germination rate, plant height at 15-day intervals, seed set, 100-seed weight, total-above-ground biomass, stomatal conductance and survival), putatively adaptive traits across the altitudinal gradient (macrohair density, pigmentation extent, and flowering time) and a neutral trait (culm diameter).  Below we list anticipated results from our common garden experiment supporting either ecotone or tension zone dynamics: \jri{i doubt culm diameter is neutral. it varies with elevation in mexicana and probably has to do with temperature. i bet a reviewer will jump on you for presuming it's neutral. also this is a TON of work, have you talked to john about the realities? stomatal conductance? Even if that takes only 15 seconds per plant and students are perfectly efficient, that's a 9 hour day just measuring conductance alone!}

\paragraph{\emph{Expected outcomes supporting ecotone dynamics}}
\begin{itemize}
\item{Plants from hybrid zones will show higher fitness in hybrid-elevation gardens}
\item{Non-admixed \emph{parviglumis} and \emph{mexicana} will show higher fitness in low- and high-elevation gardens respectively}
\item{Plants from lower elevation hybrid zone populations will have more \emph{parviglumis}-like phenotypes for putatively adaptive traits (\emph{e.g.}, fewer macrohairs, limited pigment, later flowering time), whereas plants from higher elevation hybrid zone populations will resemble \emph{mexicana} (\emph{e.g.}, highly pilose, pigmented, earlier flowering time)} \jri{not sure i follow this last one; wouldn't this be true even under a neutral scenario? closer to mexicana, look more like mexicana}
\end{itemize}

\paragraph{\emph{Expected outcomes supporting tension zone dynamics}}
\begin{itemize}
\item{Plants from hybrid zones will be less fit in all gardens (\emph{i.e.}, all environments)}
\item{Non-admixed \emph{parviglumis} and \emph{mexicana} will show higher fitness in low- and high-elevation gardens respectively} \jri{this is same as above, right? so why include it ?}
\item{Plants from hybrid populations will not show graded phenotypic variation (from \emph{parviglumis}-like to \emph{mexicana}-like) with increasing elevation. Rather, plants will possess random and maladaptive combinations of these traits} \jri{disagree, see above. just isolation by distance would predict the opposite}
\end{itemize}
		
\subsubsection*{C) Is there evidence for selection on putatively adaptive traits across hybrid zones?}

Stem pigmentation, macrohair density, and flowering time in particular are thought to be under selection in teosinte across an elevational gradient.  Pigmented and pilose plants have an advantage in retaining heat at high elevation (for a discussion of highland adaptation in the context of maize see \citealt{Eagles1994}). Additionally, \emph{mexicana} flowers much earlier than \emph{parviglumis} \citep{Rodriguez2006}, which may represent an adaptation to shorter growing seasons at high elevation. We will combine our genome-wide marker data obtained in Objective IA with phenotypic data collected in our common garden experiments in Objective IB in order to evaluate evidence for selection on these potentially adaptive traits.  A method recently developed by \citet{Ovaskainen2011} and implemented in the software DRIFTSEL \citep{Karhunen2013} is particularly suited for this purpose.  The method builds upon the $F_{ST}$--$Q_{ST}$ framework for comparison of population differentiation and quantitative trait divergence and allows the signature of selection on a given phenotypic trait to be distinguished from genetic drift.   Two sets of phenotypic data will be used for this analysis: first, data will be pooled for each population across all gardens and second, only data for populations grown in similar habitat to their collection site will be included.  Any difference in the signal of selection between these two data sets would indicate phenotypic plasticity is involved in determining phenotype.  The strength of evidence for selection for putatively adaptive traits (pigment, macrohairs, flowering time) will be compared to evidence for selection on a neutral trait (culm diameter). \jri{see comments above about neutral trait. why not do gwas if you can collect DNA from plants in field? i don't think splitting your driftsel pops up gets at plasticity the way you are thinking. or at least you don't need driftsel to tell you which traits show plasticity.}

\subsection*{Objective II: Genus-wide introgression}

\subsubsection*{Research Question IIA) Was the spread of maize facilitated by gene flow from locally-adapted wild \emph{Zea}?}  \label{adaptive_intro}

*background on spread of maize

*one new habitat was highlands

* talk about Hufford 2013: Our recently published study of \emph{mexicana}/maize hybridization in sympatric populations (Hufford et al. 2013) suggests widespread adaptive introgression from \emph{mexicana} into maize. 

* but what about spread to (insert something about environment differences) in Guatemala? 

* background on luxurians and huehue, information on maize arrival in guatemal from arch. record if available, or on maize in guat from van heerwaarden 2011.

* 4 pairs of huehue/maize and 4 pairs of lux/maize sympatric pops, then at both high and low elevation (or some other gradient?) we pick 1 allopatric teo and 1 allopatric maize, for a total of 6 pops of each taxa = 24 pops total.  

* 12 individuals per pop, GBS to 48 plex.

*D stats, g-statistic, pi <- no need for phasing

* phase with fastphase, run hapmix or other methods

* look for evidence of selection (Nielsen sweepfinder, etc.)

* can look for overlap with QTL for water-logging etc. traits (Mano papers) or GWAS for related traits in maize

* include formal analysis using Berg's SQuaT approach?

* additional questions: how much diversity remains in lux populations? what is connectivity of these pops? are lux and/or huehue threatened by introgression from maize? important to Guatemala as it is unique diversity to that country

We will also expand the scope of our previous work and assess whether similar patterns of introgression can be detected in maize populations that are sympatric to the Guatemalan teosintes (\emph{huehuetenangensis} and \emph{luxurians}).

\subsubsection*{Research Question IIB) What is the geographic scale of adaptive introgression?} \label{scale}

* in many plants local adaptation exists on a very fine geographical scale (examples, citations)

* previous work suggested introgression from mexicana allowed maize to adapt to highlands, and most of the introgression was ancient, suggesting adaptation to a broad set of challenges associated with higher elevations

* but we did see differences among populations (cite examples)

* now with increase resolution we can look at individual sympatric population pairs

* we also know parv hybridizes with maize, but unclear if any of the sympatric introgression is adaptive
* 3 pairs of mex/maize sympatric pops
* 3 pairs of parv/maize sympatric pops
* introgression, search for evidence of selection w/in individual pops

\begin{SCfigure}
  \centering
   \includegraphics[width=0.6\textwidth]{length_vs_time2}
    \caption{Effect of recombination on the expected length of a shared chromosome segment vs. number of years since divergence or introgression.  Shown are three levels of recombination roughly representing high, average, and low recombination regions of the genome.} 
\label{fig:length}
\end{SCfigure}

\subsubsection*{Research Question IIC) Did maize serve as a bridge for gene flow between previously isolated Zea taxa?} \label{bridge}

Our initial survey of divergence and gene flow in \emph{Zea}, based on a set of 26 Sanger sequence loci, found evidence for admixture between allopatric populations of \emph{mexicana} and \emph{luxurians} at multiple loci \citep{Ross-Ibarra2009a}. 
As there is no evidence to suggest that these populations overlapped in their recent history, we took these results to suggest that maize, which is known to hybridize with both taxa, may have served as a bridge for gene flow between the two teosintes.
Further support for this idea comes from genotyping of a SNP diagnostic for the two haplotypes at the inversion locus \emph{Inv4m} \citep{Fang2012,Pyhajarvi2013,Hufford2013}. 
While the inverted haplotype at this locus appears to be derived in \emph{mexicana} \citep{Pyhajarvi2013}, the SNP allele diagnostic of the inverted haplotype is nonetheless found in both maize and \emph{luxurians} samples from Guatemala (Fig. \ref{hufmap}), suggesting the possibility that the inversion has moved from \emph{mexicana} into \emph{luxurians} via a maize intermediate.
An alternative explanation, however, is that the haplotype was polymorphic in the common ancestor of both taxa, and remains segregating in each due to incomplete lineage sorting.
Simple estimates of the length of shared haplotypes expected to be unbroken by recombination suggest that, over the $sim 140,000$ divergence time between \emph{mexicana} and \emph{luxurians} \citep{Ross-Ibarra2009a}, we might well expect to see shared haplotypes of even several kb in length in low recombination regions of the genome (Fig. \ref{length}).
The high-density, genome-wide data generated here will provide an opportunity to test whether observed patterns of haplotype sharing between previously allopatric \emph{Zea} are due to recent introgression from maize.  
If shared haplotypes have come from introgression from maize over the last few thousand years, the genome-wide distribution of shared haplotype lengths should reveal longer shared segments (Fig. \ref{length}) than if haplotype sharing is due to incomplete lineage sorting alone.
Because maize has been found to hybridize with all species of \emph{Zea} \citep{Wilkes1977}, we will extend this objective to the perennial taxa \emph{Zea diploperennis} (hereafter, \emph{diploperennis}) and \emph{Zea perennis} (herafter \emph{perennis}) as well.  

We will sample 12 individuals from of 2 populations of \emph{diploperennis} and \emph{perennis} and genotype these using GBS. 
These populations, combined with samples from other teosinte populations in research questions \ref{adpative_intro} and \ref{scale}, will provide us a representative sample of wild teosinte populations from across the Americas.
We will use a number of methods, including HapMix \citep(XXX), relative nucleotide differences \citep(XXX), and STRUCTURE \citep{XXX} to compare across these sets of populations to identify shared haplotype segments. \jri{add FINE structure? something else? more detail here?}
Although we do not have a genetic map for most teosinte taxa, evidence from comparison among maize populations finds remarkable stability of the genetic map at a relatively coarse scale \citep(XXX).
We will thus use the NAM genetic map to estimate recombination rates genome-wide, and using estimates of divergence time between taxa, generate an expected distribution of shared haplotype lengths.
This expected distribution will be compared to that observed in pairwise comparisons between taxa to look for evidence of recent introgression.
We will also scan a large sample of $>16,000$ maize samples that have been genotyped with GBS (www.panzea.org) to look for the presence of such haplotypes in domesticated maize.

\section*{Broader Impacts}

Our efforts to broaden the impact of the research proposed here will begin within our groups through our commitment to effectively mentor volunteer undergraduate interns as well as graduate students and/or postdoctoral scholars funded by the project. Students and postdocs will receive one-on-one training from the investigators and senior personnel on laboratory, computational, and field research methods.  Mentees will also be encouraged and funded to present their work at scientific conferences.  Our groups have an excellent mentoring track record with four undergraduate students in the last five years publishing their work in scholarly journals and multiple underrepresented minorities participating in our research.

\subsection*{ISU GK12 Fellowship Program}
	
In addition to the student and postdoc mentoring that will occur within our groups, as part of our broader impact activities each year one of our graduate students will participate in Iowa State University's GK12 Fellowship program: \emph{Symbi},  \url{http://www.gk12.iastate.edu/default.asp}. The selected graduate student will spend one full day each week in a middle or high school science classroom for the entire academic year of the Des Moines Public School District. This is the largest and most diverse school district in Iowa with over 50\% underrepresented minority student enrollment and over 70\% of students receiving free or reduced-cost lunch. The graduate student will introduce the K12 students to the scientific process through inquiry-based activities, relate the students� science curriculum to real world examples, work with students on their science fair projects, and serve as a role model in a STEM profession. Furthermore, the graduate student will introduce students to his/her research project on hybridization and introgression in \emph{Zea}, a topic that is particularly well suited for teaching evolution in Iowa given the important role that maize plays in the Iowan economy. In introducing his/her dissertation research, the graduate student will engage Des Moines students in how research is conducted and provide STEM content professional training to his/her partner teacher. The GK12 Fellow will work with approximately 150 students on a regular weekly basis.  Student assessments from this program have shown that a significant number of students like science more after having a GK Fellow in the classroom.  Teachers report that having a GK12 Fellow in their classroom is excellent professional development.  The PI will also visit the classroom and will support the selected graduate student in their development of appropriate material for the K12 audience.

\subsection*{US-Mexico Exchange Program}
	
Finally, we will establish a student exchange program between the Eguiarte Laboratory at UNAM in Mexico and the Hufford and Ross-Ibarra Laboratories in the United States. The Ross-Ibarra Laboratory has run an NSF-supported, US-Mexico exchange program for the last three years.  All of the exchange students involved in the program have continued on to additional graduate work, and two have earned authorship on forthcoming papers from their internship.  We will build upon the success of this program.  A student from the Eguiarte group will spend 2-3 months in either the Hufford or Ross-Ibarra Laboratory learning the GBS methodology and/or honing his/her skills in population genomic analysis, whereas a student from the Hufford and/or Ross-Ibarra Laboratories will travel to Mexico to participate in sample collection trips and to obtain expertise in common garden field experiments. This exchange will build capacity in all groups involved and will provide a valuable international research experience for a graduate student supported by the grant.  

Senior Personnel Claudia Calderon has previously led international student research trips and will assist in preparing students from both the United States and Mexico for the exchange program. A survey will be given to both exchange students and faculty in order to gauge expectations prior to the trip and facilitate collaborations amongst the labs.  The survey will also assess students' knowledge and preconceived ideas  regarding their travel destinations.  A meeting (online or face-to-face) with the cohort of students traveling will help address these pre-conceptions and reduce cultural misunderstandings.  Suggestions will be given to students of how to prepare before the trip (visa, immigration requirements) and how to communicate with their peers and others during their exchange.  Students will be given information regarding the facilities where they will be staying, transportation to be used, food and water safety, the availability of telecommunications and general safety guidelines.

\required{Results From Prior NSF Support}
% 5 pages or fewer of the 15 pages for entire description document.
% include results from NSF grants received in the past 5 years.
% If supported by more than one grant, choose the most relevant one.

% For each grant, include: 
%	(a) NSF award number, amount, dates of support 
%	(b) The title of this project
%	(c) Publications resulting from this research
%	(d) Summary of the results of the completed work
%	(e) A brief description of data samples available and other research products not described 	      elsewhere
%	(f) For renewed support, a description of the relationship between the completed and 			      proposed work

% Due to space limitations, it is often advisable to use citations rather
% than putting the titles of the publications in the body 
% of this section

\subsection*{Ross-Ibarra: \#1238014: Biology of Rare Alleles in Maize and Its Wild Relatives}
\$13,311,185 (\$2,368,767 to Ross-Ibarra and \$1,206,211 to Flint-Garcia), 05/15/13-04/30/18. PI Edward Buckler, co-PIs J. Doebley, J. Holland, S. Flint-Garcia, Q. Sun, P. Bradbury, S. Mitchell, J. Ross-Ibarra
\par\noindent{\bf Intellectual merit} In the first year we have developed accurate imputation approaches, found evidence for the importance of deleterious variants and non-genic polymorphisms in heterosis and GWAS, documented differences in recombination among the parents of the NAM population, and found population genetic evidence suggesting the importance of demography and purifying selection across the genome.  The grant has produced 18 total publications in its first year (only publications involving PIs Flint-Garcia and Ross-Ibarra are shown below). 
\par\noindent{\bf Broader impacts}  In the first year this project has included 10 postdoctoral and 12 graduate trainees. The GBS workshop and traveling maize exhibit continue to be popular and successful. A new version of the teacher-friendly guide to maize evolution has been revised and published online. 
\par\noindent{\bf Publications} \citet{peiffer2013genetic, Romay2013, wills2013many, Mezmouk2014, Peiffer2014, sood2014mining}

\subsection*{Ross-Ibarra: \#0922703: Functional Genomics of Maize Centromeres}
\$5,008,031 (\$754,409 to Ross-Ibarra). 09/01/09-08/31/14. PI Kelly Dawe, co-PIs J. Birchler, J. Jiang, G. Presting, J. Birchler, J. Ross-Ibarra
\par\noindent{\bf Intellectual merit} Centromeres are regions of the genome that organize and regulate chromosome movement, yet the biology of centromeres remains poorly understood. Co-PI Ross-Ibarra's group has focused in particular on the evolutionary genetics of centromeres. This work has demonstrated the remarkable evolutionary lability of centromere tandem repeats, but has shown that there is little evidence in maize for coevolution between centromere sequence and kinetochore proteins. Ongoing work from the Ross-Ibarra lab seeks to characterize kinetochore proteins, assess the phylogenetic evidence for longer-term coevolution, and understand patterns of centromere and genome size variation in natural populations.
\par\noindent{\bf Broader impacts}  Co-PI Ross-Ibarra has established an international student exchange program as part of this grant. Data and result of this project have been disseminated via publications and presentations as well as deposited in the maize genetics community database \url{www.maizegdb.org}. Former trainees on the grant include Dr. Matthew Hufford (Co-PI on the current grant). 
\par\noindent{\bf Publications} \citet{Shi2010a, Chia2012a, Fang2012, Hufford2012, Hufford2012b, Hufford2013, Melters2013a, Kanizay2013, Pyhajarvi2013}

