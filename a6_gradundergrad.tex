\renewcommand{\thepage}{Plans for Undergraduate and Graduate Student Mentoring  - Page \arabic{page} of 1}
\required{Supplementary Documentation}
\required{Plans for Undergraduate and Graduate Student Mentoring }

\subsection*{Undergraduate Students}

Only Iowa State has requested funding for undergraduate students, but it is anticipated that undergraduate students will participate in unfunded internship roles at UC Davis and possibly USDA-ARS through the University of Missouri. Undergraduates will  be partnered directly with a graduate student or postdoc. Unpaid undergraduate interns will be expected to develop specific research projects, and are expected to present on the progress of their work during regular group meetings.
In addition to research experience in the lab or in the field, undergraduates will be encourage to attend regular lab meetings, and lab journal clubs; this is already regularly the case for students working with Drs. Ross-Ibarra and Coop.  UC Davis undergraduates have also presented their work at university-sponsored research conferences and numerous students have earned authorship on peer-reviewed publications.  Students will be given opportunities to develop data analysis and management skills, both through the field management system of Dr. Flint-Garcia, and through learning basic statistical and bioinformatics tools such as R and Unix at UC Davis or Iowa State. 
Undergraduate students will be provided guidance about potential careers in biology and plant science (see, for example, \url{http://www.slideshare.net/jrossibarra/forgradschool}).  

\subsection*{Graduate Students}

The current proposal requests funding for graduate students only at UC Davis, although it is hoped that additional students will participate in this grant through other funding mechanisms (institutional support, competitive fellowships, etc.). Students will be trained in order to prepare them for research careers (academic or otherwise).  All students will be expected to take part in internal lab meetings (the Coop and Ross-Ibarra labs at UC Davis hold joint lab meetings) at which pthey will be given numerous opportunities to hone their presentation skills.  The Coop, Ross-Ibarra and Hufford labs currently host weekly journal clubs in which students gain additional training in reading, presenting, and dissecting scientific literature. All members of the Ross-Ibarra and Flint-Garcia labs also attend a weekly journal club as part of another collaborative project (NSF \#1238014). In addition, we will organize a monthly group meeting via web-conference in which one lab member presents on their research progress.  UC Davis has a ReadyTalk license allowing inexpensive web-conference hosting. All of our institutions have seminar series specifically for postdoctoral and graduate students to practice presentation skills; members of our labs will be encouraged to attend these.
Graduate students on the grant will be expected to produce first-author papers for peer-review as part of their project, and encouraged to contribute to additional papers as middle author.  Students will be expected to attend and present a poster or talk at a scientific conference each year; UC Davis provides several opportunities for travel funds to support students in this manner. Finally, issues of ethics and organization will be included in training.  These will include authorship, reproducibility, and basic scientific ethics. For example students will be encourage to pursue open science, including the submission of preprints and pre-publication data release and students will be required to maintain Github repositories of their computational work to ensure reproducibility and transparency.

 