\renewcommand{\thepage}{Postdoctoral Researcher Mentoring Plan - Page \arabic{page} of 1}
\required{Supplementary Documentation}
\required{Postdoctoral Researcher Mentoring Plan}

%\emph{Maximum of 1 page}
%Each proposal that requests funding to support postdoctoral researchers must include, as a supplementary document, a description of the mentoring activities that will be provided for
%such individuals. In no more than one page, the mentoring plan must describe the
%mentoring that will be provided to all postdoctoral researchers supported by
%the project, irrespective of whether they reside at the submitting organization,
%any subawardee organization, or at any organization participating in a
%simultaneously submitted collaborative project. Proposers are advised that the
%mentoring plan may not be used to circumvent the 15-page project description
%limitation.

%\smallskip
%Examples of mentoring activities include, but are not limited to: career
%counseling; training in preparation of grant proposals, publications and presentations;
%guidance on ways to improve teaching and mentoring skills; guidance on how to
%effectively collaborate with researchers from diverse backgrounds and disciplinary areas;
%and training in responsible professional practices. The proposed mentoring activities
%will be evaluated as part of the merit review process under the broader impacts merit
%review criterion. Proposals that include funding to support postdoctoral researchers
%and do not include the requisite mentoring plan will be returned without review.

%\bigskip
%\textbf{Documentation of collaborative arrangements} of significance to the
%proposal through letters of commitment.

%\smallskip
%Proposers are reminded that, unless required by a specific program solicitation,
%letters of support should not be submitted as they are not a standard component
%of an NSF proposal, and, if included, a reviewer is under no obligation to review
%these materials. Letters of support submitted in response to a program solicitation
%requirement must be unique to the specific proposal submitted and cannot be altered
%without the author's explicit prior approval. NSF may return without review proposals
%that are not consistent with these instructions.

%\bigskip
%\textbf{Documentation regarding research involving the use of human subjects}, hazardous
%materials, vertebrate animals, or endangered species.

The current proposal requests funding for one postdoctoral researchers at UC Davis. We also hope additional postdocs may join the group via alternative funding opportunities (fellowships, etc.) and anticipate that postdocs funded on other grants may collaborate to a greater or lesser degree on this project.  Much of our thinking on postdoctoral mentoring comes directly from our own mentorship experience -- PIs Hufford and Ross-Ibarra were both postdoctoral scholars on NSF-funded programs. For this project, each PI will act as mentor and supervisor for postdocs in their lab, holding regular weekly meetings to assess progress and set goals.  One clear goal will be first authorship on submitted papers, with the expectation of approximately one first author paper per year of duration of the postdoc. 

Interaction and experience presenting and discussing science will be highly encouraged. All groups will have internal lab meetings at which postdocs and graduate students will be given numerous opportunities to hone their presentation skills.  Both the Ross-Ibarra and Hufford labs currently host weekly journal clubs in which postdocs gain additional training in reading, presenting, and dissecting scientific literature. Members of the Ross-Ibarra lab also commonly write blog post critiques of the papers read in journal club, on occasion eliciting written response from the authors.  This provides excellent training in reviewing and in scientific communiciation.  Members of both labs also attend a weekly journal club as part of another collaborative project (NSF \#1238014). In addition, we will organize a monthly group meeting via web-conference in which one lab member presents on their research progress.  UC Davis has a ReadyTalk license allowing inexpensive web-conference hosting. Both institutions have seminar series specifically for postdoctoral and graduate students to practice presentation skills; members of our labs will be encouraged to attend these.

Another important aspect of training will be experience mentoring graduate students and undergraduates.  Postdocs will be given the opportunity to supervise undergraduate and/or graduate students on projects related to the grant.  Previous efforts to encourage such supervision in our labs have been very successful, with postdoc-mentored students presenting conference posters on their research or earning authorship on papers.  Lab alumni have confirmed the utility of supervisory experience in applying for jobs, especially in industry.

Postdocs will be encouraged to write and apply for external funding, including fellowships and grant proposals.  The Ross-Ibarra lab has a documented history of successful funding with postdoctoral scholars as Co-PIs, providing valuable training (and even initial funding) for the scholars' future academic careers.

Finally, postdocs will be encouraged to take advantage of professional development programs offered by their local institutions. All of our institutions have infrastructure in place for professional development of postdocs and offer training in responsible conduct of research, grantsmanship, mentoring, career development, authorship of journal papers, and teaching. 


